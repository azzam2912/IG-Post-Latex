\documentclass{article}
\usepackage[document]{ragged2e}

\usepackage{enumitem}
\renewcommand{\baselinestretch}{1.5}
\addtolength{\oddsidemargin}{-1in}
\addtolength{\evensidemargin}{-1.5in}
\addtolength{\textwidth}{1.9in}

\addtolength{\topmargin}{-1in}
\addtolength{\textheight}{3in} 

\usepackage{amsmath, amsthm}
\usepackage{amssymb}

\renewcommand{\qed}{\unskip\nobreak\quad\qedsymbol}% As suggested by egreg

\makeatletter
\newtheoremstyle{indented}
{10pt}% space before
{10pt}% space after
{\addtolength{\@totalleftmargin}{3.5em}
	\addtolength{\linewidth}{-3.5em}
	\parshape 1 5em \linewidth}% body font
{}% indent
{\bfseries}% header font
{.}% punctuation
{.5em}% after theorem header
{}% header specification (empty for default)
\makeatother

\theoremstyle{indented}

\newtheorem*{lemma}{Lemma}


\begin{document}
	Didefinisikan barisan $(x_n)$ dan $(y_n)$ dimana $x_1 = 3, y_1 = 2,$ serta $$x_{n+1}=x_n+2y_n, \; \; y_{n+1}=x_n+y_n, \; \; \forall n \in \mathbb{N}.$$
	
	Misalkan sebuah barisan $(z_n)$ didefinisikan dengan $$z_n = \frac{x_n}{y_n}, \; \; \forall n \in \mathbb{N}.$$
	
	\begin{enumerate}
		\item[ a.] Buktikan bahwa $|z_{n+1}-\sqrt{2}| < \frac12|z_n - \sqrt{2}| \forall n \in \mathbb{N}$
		\item[ b.] Carilah $\lim (z_n)$
	\end{enumerate}

	(Seleksi KNMIPA 2021 tingkat UI)\\
	\newpage
	$\textbf{Solusi. }$
	
	\begin{enumerate}
		\item[a.] Perhatikan bahwa untuk $n>0$,
		$$
		z_{n+1} = \frac{x_{n+1}}{y_{n+1}} 
		= \frac{x_n +  2y_n}{x_n+y_n} 
		= 1 + \frac{y_n}{x_n+y_n} 
		= 1 + \frac{1}{1+\frac{x_n}{y_n}} 
		= 1 + \frac{1}{1 + z_n}$$
		Sekarang, perhatikan lemma berikut
		 
		 \begin{lemma} $z_n > 1$ untuk $\forall n \in \mathbb{N}$.
		 		
		 		\begin{proof}[Bukti.]
		 			Untuk $n=1$, didapat $z_1 = \dfrac{x_1}{y_1} = \dfrac{3}{2} > 1$.\\
		 			Asumsikan pernyataan benar untuk $n=k > 0$. Maka untuk $n = k+1$, didapat 
		 			\begin{align*}
		 				z_{k+1}
		 				= 1 + \frac{1}{1 + z_k}
		 				> 1 + 0 = 1
		 			\end{align*}
		 			Sehingga, berdasarkan prinsip induksi, terbukti bahwa $z_n > 1, \; \forall n \in \mathbb{N}$.
		 		\end{proof}
		 		
		 	\end{lemma}
		 
		
		dari lemma tersebut, untuk $\forall n \in \mathbb{N}$, kita punya
		\begin{align*}
			\left|z_{n+1} - \sqrt{2}\right| 
			&= \left|1 - \sqrt{2} + \frac{1}{1+z_n}\right|  \\
			&= \left|\frac{z_n-\sqrt{2}z_n+2-\sqrt{2}}{z_n+1}\right| \\
			&= \frac{1}{\left|z_n+1\right|}\left|1-\sqrt{2}\right|\left|z_n-\sqrt{2}\right| \\
			&< \frac{1}{1+1}\cdot1\cdot\left|z_n-\sqrt{2}\right| \\
			&= \frac12\left|z_n-\sqrt{2}\right|
		\end{align*}
	soal bagian (a) terbukti. \qed
	\newpage
	\item[b.] Akan ditunjukkan bahwa $\lim (z_n) = \sqrt{2}$
	\begin{lemma}
		$\left|z_{n} - \sqrt{2}\right| \le \frac{1}{2^{n-1}}\left|z_1-\sqrt{2}\right|$ untuk $n \ge 1$
		\begin{proof}
			Untuk $n=2$, didapat $\left|z_{1} - \sqrt{2}\right|
			\le \frac{1}{2^{1-1}}\left|z_1-\sqrt{2}\right|$ (benar).\\
			Asumsikan pernyataan benar untuk $n=k$. Maka untuk $n = k+1$, didapat 
			\begin{align*}
				\left|z_{k+1} - \sqrt{2}\right|
				< \frac12\left|z_k-\sqrt{2}\right| <\frac12 \cdot \frac{1}{2^{k-1}}\left|z_1-\sqrt{2}\right| = \frac{1}{2^{k}}\left|z_1-\sqrt{2}\right|
			\end{align*}
			Sehingga, berdasarkan prinsip induksi, pernyataan terbukti.
		\end{proof}
	\end{lemma}
	
	dari lemma tersebut kita punya $$0 \le \left|z_{n} - \sqrt{2}\right| \le \frac{1}{2^{n-1}}\left|z_1-\sqrt{2}\right|.$$
	
	Padahal karena $$\lim\limits_{n \rightarrow \infty}\frac{1}{2^{n-1}}\left|z_1-\sqrt{2}\right| = 0,$$
	
	dengan teorema Squeeze, kita mendapatkan
	$$\lim\limits_{n \rightarrow \infty} 0 = 0\le \lim\limits_{n \rightarrow \infty}\left|z_n-\sqrt{2}\right| \le \lim\limits_{n \rightarrow \infty} \frac{1}{2^{n-1}}\left|z_1-\sqrt{2}\right| = 0$$
	$$\implies \lim\limits_{n \rightarrow \infty}\left|z_n-\sqrt{2}\right| = 0$$
	$$ \implies \lim\limits_{n \rightarrow \infty}z_n = \sqrt2$$
	
	sehingga terbukti bahwa $\lim (z_n) = \sqrt{2}$. \qed
	\end{enumerate}
	
	
\end{document}
