\section{Soal}
Pada segitiga $ABC$, titik $D$ adalah kaki garis tinggi dari $C$. Titik $E$ dan $F$ pada $AC$ dan $BC$, berturut-turut $AE = AD$ dan $BF = BD$. Titik $S$ adalah refleksi $C$ pada titik pusat lingkaran luar $\triangle ABC$. Buktikan bahwa $SE=SF$.
\begin{center}
\begin{asy}
import olympiad;
import geometry;
unitsize(1.3cm);
pair A,B,C,D,S,O,E,F;
A = (-3,0);
B = (3,0);
C = (-1,5);
D = foot(C,A,B);
O = circumcenter(A,B,C);
S = rotate(180,O)*C;
E = intersectionpoints(line(A,C),Circle(A,abs(A-D)))[0];
F = intersectionpoints(line(B,C),Circle(B,abs(B-D)))[0];
draw(A--B--C--cycle);
draw(C--D);
draw(C--S);
draw(E--D--F);
draw(A--S--B);
draw(E--S--F, dashed+red);
draw(Circle(O,0.05),red);
draw(circumcircle(triangle(A,B,C)));
draw(rightanglemark(B,D,C,12), purple);
draw(rightanglemark(S,A,C,12), blue);
draw(rightanglemark(C,B,S,12), blue);

label("$A$", A, SW);
label("$B$", B, SE);
label("$C$", C, N);
label("$D$", D, S);
label("$O$", O, NE);
label("$E$", E, NW);
label("$F$", F, NE);
label("$S$", S, SE);
\end{asy}
\end{center}

% \newpage
% \section{Walkthrough}
% Sadari kalau $CS$ adalah diameter lingkaran. Dari sini akan didapat banyak sudut siku-siku. Banyak sudut siku-siku berarti banyak yang bisa di Pythagoras atau pakai trigonometri. Lalu, kalau mau melakukan length chasing juga bisa pakai cara nguli-nguli dengan dalil cosinus.


\newpage
\section{Solusi 1 (Length Chasing - Pythagoras)}

Perhatikan bahwa $CS$ adalah diameter lingkaran luar $\triangle ABC$ sehingga $\angle CAS = \angle SBC = 90^\circ$. Oleh karena itu, dengan teorema Pythagoras di segitiga $EAS, SBF, CAS, CSB$ akan didapat
\begin{align*}
    CS^2 &= CS^2\\
    AC^2 + AS^2 &= BC^2+BS^2\\
    AC^2 - CD^2 + AS^2 &= BC^2 - CD^2 + BS^2\\
    AD^2 + AS^2 &= BD^2 + BS^2\\
    AE^2 + AS^2 &= BF^2 + BS^2\\
    SE^2 &= SF^2\\
    SE &= SF.
\end{align*}
Terbukti.

\newpage
\section{Solusi 2 (Sedikit Trigonometri - Dalil Cosinus)}
Perhatikan bahwa $\angle SAC = \angle CBS = 90^\circ$ karena $CS$ diameter lingkaran $(ACBS)$. Dari sini didapat 
$$\cos \angle SCF = \cos \angle SCB = \frac{BC}{CS}$$ 
dan 
$$\cos \angle ECS = \cos \angle ACS = \frac{AC}{CS}.$$

Dengan dalil cosinus pada $\triangle FCS$ serta teorema Pythagoras di $\triangle CDB$ didapat
\begin{align*}
    SF^2 &= CF^2 + CS^2 - 2\cdot CF \cdot CS \cos \angle SCF\\
    SF^2 &= CF^2 + CS^2 - 2\cdot CF \cdot BC\\
    SF^2 &= (BC-CF)^2 - BC^2 + CS^2\\
    SF^2 &= BF^2 - BC^2 + CS^2\\
    SF^2 &= BD^2 - BC^2 + CS^2\\
    SF^2 &= -CD^2 + CS^2
\end{align*}
Selanjutnya, dengan dalil cosinus pada $\triangle ECS$, serta teorema Pythagoras di $\triangle ADC$ didapat
\begin{align*}
    SE^2 &= CE^2 + CS^2 - 2\cdot CE \cdot CS \cos \angle ECS \\
    SE^2 &= CE^2 + CS^2 - 2\cdot CE \cdot AC\\
    SE^2 &= (AC-CE)^2 - AC^2 + CS^2\\
    SE^2 &= AE^2 - AC^2 + CS^2\\
    SE^2 &= AD^2 - AC^2 + CS^2\\
    SE^2 &= -CD^2 + CS^2\\
    SE^2 &= SF^2\\
    SE &= SF
\end{align*}
Terbukti.

\newpage
\section{Solusi 3 (Trigonometrinya banyak)}
Perhatikan bahwa $\angle SAC = \angle CBS = 90^\circ$ karena $CS$ diameter lingkaran $(ACBS)$. Lalu, didapat $\angle BSC = \angle BAC = A$ dan $\angle CSA = \angle CBA = B$ sehingga
\begin{align*}
    SB &= \dfrac{BC}{\tan \angle BSC} = \dfrac{BC}{\tan A}\\
    SA &= \dfrac{AC}{\tan \angle CSA} = \dfrac{AC}{\tan B}
\end{align*}
Selanjutnya, observasi $\triangle BDC$ dan $\triangle ADC$ akan didapat
\begin{align*}
    BF &= BD = BC \cos B\\
    AE &= AD = AC \cos A
\end{align*}
Oleh karena itu dengan teorema Pythagoras
\begin{align*}
    SF^2 &= SB^2 + BF^2 = \dfrac{BC^2}{\tan^2 A} + BC^2 \cos^2 B\\
    SE^2 &= SA^2 + AE^2 = \dfrac{AC^2}{\tan^2 B} + AC^2 \cos^2 A
\end{align*}
Sekarang dengan properti trigonometri dan dalil sinus pada $\triangle ABC$ kita punya
\begin{align*}
    1 &= 1\\
    \dfrac{\cos^2 A}{\cos^2 A} &= \dfrac{\cos^2 B}{\cos^2 B}\\
    \dfrac{1-\sin^2 A}{\cos^2 A} &= \dfrac{1-\sin^2 B}{\cos^2 B}\\
    \dfrac{1}{\cos^2 A}+\dfrac{\sin^2 B}{\cos^2 B} &= \dfrac{1}{\cos^2 B}+\dfrac{\sin^2 A}{\cos^2 A}\\
    \dfrac{1}{\cos^2 A}+\tan^2 B &= \dfrac{1}{\cos^2 B}+\tan^2 A\\
    1 &= \dfrac{\frac{1}{\cos^2 A}+\tan^2 B}{\frac{1}{\cos^2 B}+\tan^2 A}\\
    \left(\dfrac{\sin A}{\sin B}\right)^2 &= \dfrac{\sin^2 A\left(\frac{1}{\cos^2 A}+\tan^2 B\right)}{\sin^2 B\left(\frac{1}{\cos^2 B}+\tan^2 A\right)}\\
    \left(\dfrac{BC}{AC}\right)^2 &= \dfrac{\tan^2 A + \sin^2 A \tan^2 B}{\tan^2 B + \sin^2 B \tan^2 A}\\
    BC^2(\tan^2 B + \sin^2 B \tan^2 A) &= AC^2(\tan^2 A + \sin^2 A \tan^2 B)\\
    \dfrac{BC^2(\tan^2 B + \sin^2 B \tan^2 A)}{\tan^2 A \tan^2 B} &= \dfrac{AC^2(\tan^2 A + \sin^2 A \tan^2 B)}{\tan^2 A \tan^2 B}\\
    \dfrac{BC^2}{\tan^2 A} + BC^2 \cos^2 B &= \dfrac{AC^2}{\tan^2 B} + AC^2 \cos^2 A\\
    SF^2 &= SE^2\\
    SF &= SE. \qed
\end{align*}