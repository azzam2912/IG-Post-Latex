\documentclass[12pt]{scrartcl}
\usepackage[sexy]{evan}
\usepackage{graphicx,amsmath,amssymb,amsthm,amsfonts,babel}
\usepackage{tikz, tkz-euclide}
\usetikzlibrary{calc,through,intersections}
\usepackage[paperwidth=16cm, paperheight=16cm,margin=1.5cm]{geometry}

\title{KTOM November 2021 Esai Nomor 2 }
\author{Edisi Kontes Akhir Tahun (Bagian 1)}
\date{Official solution by Azzam L. H.}

\begin{document}
\maketitle
\pagestyle{plain}
\noindent
 \textbf{Soal.} Diberikan sebuah segitiga lancip $ABC$ dengan titik tinggi $H$ dan lingkaran dalamnya menyinggung $BC$ di $D$. Misalkan titik $X$ di dalam segitiga $ABC$ sedemikian sehingga lingkaran dalam $\triangle BXC$ menyinggung $BC$ di $D$ dan $B,H,X,C$ berada dalam satu lingkaran, sebut sebagai lingkaran $\Gamma$.  Sekarang misalkan garis bagi dalam $\angle BXC$ memotong lingkaran $\Gamma$ di $T$. Misalkan pula $AM$ adalah garis berat $\triangle ABC$ dengan $AM$ memotong lingkaran $\Gamma$ di titik $N$ yang terletak di luar $\triangle ABC$. Jika $L$ adalah perpotongan antara $HN$ dan $MT$, buktikan bahwa $HL=LN$.	
 \newpage	

 \usetikzlibrary{arrows}
 \pagestyle{empty}
 
 \definecolor{uuuuuu}{rgb}{0.26666666666666666,0.26666666666666666,0.26666666666666666}
 \definecolor{ududff}{rgb}{0.30196078431372547,0.30196078431372547,1}
 
 \begin{center}
 \begin{tikzpicture}[line cap=round,line join=round,>=triangle 45,x=1cm,y=1cm,scale=0.55]
 \draw [line width=1pt] (-2.1089294063040342,6.607463870020379)-- (-3.1937593138204754,-4.304648729115588);
 \draw [line width=1pt] (-3.1937593138204754,-4.304648729115588)-- (7.686446523329714,-3.8898608233004777);
 \draw [line width=1pt] (7.686446523329714,-3.8898608233004777)-- (-2.1089294063040342,6.607463870020379);
 \draw [line width=1pt] (2.428567522062273,-8.877128299431872) circle (7.246939248430825cm);
 \draw [line width=1pt] (-3.1937593138204754,-4.304648729115588)-- (0.46549660012814614,-1.9011351066381876);
 \draw [line width=1pt] (0.46549660012814614,-1.9011351066381876)-- (7.686446523329714,-3.8898608233004777);
 \draw [line width=1pt] (-1.7444815716887268,-2.9522831764272994)-- (6.601616615813272,-14.801973422436443);
 \draw [line width=1pt] (2.2463436047546192,-4.0972547762080325)-- (2.7046432503354554,-16.11880701798226);
 \draw [line width=1pt] (0.46549660012814614,-1.9011351066381876)-- (2.7046432503354554,-16.11880701798226);
 \draw [line width=1pt] (-2.1089294063040342,6.607463870020379)-- (6.601616615813272,-14.801973422436443);
 \draw [line width=1pt] (7.686446523329714,-3.8898608233004777)-- (6.601616615813272,-14.801973422436443);
 \draw [line width=1pt] (-3.1937593138204754,-4.304648729115588)-- (6.601616615813272,-14.801973422436443);
 \draw [line width=1pt] (-1.7444815716887268,-2.9522831764272994)-- (7.686446523329714,-3.8898608233004777);
 \draw [line width=1pt] (-1.7444815716887268,-2.9522831764272994)-- (-3.1937593138204754,-4.304648729115588);
 \begin{scriptsize}
 \draw [fill=ududff] (-2.1089294063040342,6.607463870020379) circle (2.5pt);
 \draw[color=ududff] (-1.7254607548777698,7.5922975735286045) node {$A$};
 \draw [fill=ududff] (-3.1937593138204754,-4.304648729115588) circle (2.5pt);
 \draw[color=ududff] (-2.8252870951821243,-3.3059816167599907) node {$B$};
 \draw [fill=ududff] (7.686446523329714,-3.8898608233004777) circle (2.5pt);
 \draw[color=ududff] (8.072992095106478,-2.906044765740226) node {$C$};
 \draw [fill=uuuuuu] (-1.7444815716887268,-2.9522831764272994) circle (2pt);
 \draw[color=uuuuuu] (-1.3755160102354753,-2.056178957323225) node {$H$};
 \draw [fill=uuuuuu] (0.5516826167789406,-4.161860620324407) circle (2pt);
 \draw[color=uuuuuu] (0.9241208831281746,-3.2559895103825203) node {$D$};
 \draw [fill=uuuuuu] (0.46549660012814614,-1.9011351066381876) circle (2pt);
 \draw[color=uuuuuu] (0.8241366703732333,-1.006344723396342) node {$X$};
 \draw [fill=uuuuuu] (2.2463436047546192,-4.0972547762080325) circle (2pt);
 \draw[color=uuuuuu] (2.623852499962177,-3.2059974040050494) node {$M$};
 \draw [fill=uuuuuu] (6.601616615813272,-14.801973422436443) circle (2pt);
 \draw[color=uuuuuu] (6.973165754802124,-13.904308168783759) node {$N$};
 \draw [fill=uuuuuu] (2.7046432503354554,-16.11880701798226) circle (2pt);
 \draw[color=uuuuuu] (3.0737814573594124,-15.204102934597996) node {$T$};
 \draw [fill=uuuuuu] (2.4285675220622727,-8.87712829943187) circle (2pt);
 \draw[color=uuuuuu] (2.8238209254720594,-7.9552475098647575) node {$L$};
 \end{scriptsize}
 
 \end{tikzpicture}
 \end{center}
 
  \begin{proof}[\textbf{Solusi.}]
  Notasikan $\measuredangle$ sebagai sudut berarah atau \textit{directed angle} $\mod 180^\circ$.
  Perhatikan karena $H$ titik tinggi $\triangle ABC$ dan $BHCN$ siklis, maka $$\measuredangle BNC = \measuredangle BHC = \measuredangle HBC + \measuredangle BCH = \measuredangle CAH + \measuredangle HAB = \measuredangle CAB.$$ Lalu, karena $M$ titik tengah $BC$, maka diperoleh bahwa $N$ adalah hasil pencerminan $A$ terhadap $M$. Dari sini $\triangle ABC \cong \triangle NCB$ sehingga $\measuredangle BCN = \measuredangle CBA$ yang menyebabkan
 $$\measuredangle HCN = \measuredangle HCB + \measuredangle BCN = \measuredangle HCB + \measuredangle CBA = 90^\circ.$$
 Berarti didapat bahwa $HN$ merupakan diameter lingkaran $\Gamma$. 
 
 Sekarang, perhatikan bahwa panjang  busur $BT$ sama dengan panjang busur $TC$ karena $XC$ merupakan garis bagi $\angle BXC$, yang berarti $BT=TC$. Namun, kita tahu bahwa $BM=MC$. Oleh karena itu, didapat $TM \perp BC$.
 
 Padahal, karena $BC$ adalah tali busur $\Gamma$, maka haruslah $MT$ berhimpit dengan diameter lingkaran $\Gamma$ yang dengan kata lain, $MT$ melalui titik pusat lingkaran $\Gamma$ yang juga merupakan titik tengah $HN$. Terbukti bahwa $HL = LN$.
 \end{proof}


\end{document}