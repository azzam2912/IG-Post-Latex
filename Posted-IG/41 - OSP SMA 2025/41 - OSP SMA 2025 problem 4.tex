\section{Soal - OSP SMA 2025 Nomor 4 Isian Singkat}
Misalkan bilangan asli $a, b, c, d$ memenuhi persamaan $$2^a+2^b+2^c=4^d.$$
    Jika $a+b+c+d \le 500$, maka nilai terbesar\\ yang mungkin dari $d$ adalah\dots

\newpage
\section{Motivasi}
Pas lihat pertama kali ada banyak 2 pangkat dan ada 4 pangkat yang sebenarnya juga 2 pangkat \textit{in disguise} (inget aja kalo $4 = 2^2$). Nah kalo udah ada kumpulan bilangan $n$ pangkat terus dijumlahin $\implies$ bilangan itu bisa diubah ke basis-$n$. Dari sini tinggal operasikan bilangan biner deh, gampang =D

\section{Prolog}
\textbf{Definisi}
(\textit{Catatan: Definisi ini diambil dari hasil generasi Gemini Pro 2.5 :D})\\
Sistem bilangan biner merepresentasikan nilai numerik menggunakan dua simbol atau digit unik, yaitu 0 dan 1. Setiap digit dalam sistem ini disebut sebagai bit (singkatan dari \textit{binary digit}).

Representasi Matematis
Sebuah bilangan biner yang direpresentasikan sebagai barisan bit $(d_k d_{k-1} \dots d_1 d_0)_2$, di mana setiap $d_i \in \{0, 1\}$, memiliki nilai ekuivalen dalam basis 10 (desimal) yang dihitung menggunakan formula berikut:
$$V_{10} = \sum_{i=0}^{k} d_i \cdot 2^i = (d_k \cdot 2^k) + (d_{k-1} \cdot 2^{k-1}) + \dots + (d_1 \cdot 2^1) + (d_0 \cdot 2^0)$$

Sebagai contoh, bilangan biner $(1101)_2$ merepresentasikan nilai desimal:$$(1 \cdot 2^3) + (1 \cdot 2^2) + (0 \cdot 2^1) + (1 \cdot 2^0) = 8 + 4 + 0 + 1 = 13$$

\section{Solusi}
Perhatikan bahwa untuk sembarang bilangan bulat non-negatif $n$, representasi biner (basis 2) dari $2^n$ adalah $1\underbrace{000...00_2}_{n \text{ buah } 0}$. Lalu, diketahui juga lemma yang sangat kecil berikut 
\begin{lemma*}
    Misal $x,y$ adalah dua bilangan bulat non-negatif.
    
    Jika $x=y=n$ maka
    $$2^n + 2^n = 2 \cdot 2^n = 2^{n+1} \text{ yang berarti } 1\underbrace{000...000_2}_{n \text{ buah } 0} + 1\underbrace{000...000_2}_{n \text{ buah } 0} = 1\underbrace{000...000_2}_{n+1 \text{ buah } 0}.$$
    
    Jika $x \neq y$, WLOG $x > y$ maka 
    $$1\underbrace{000...0000_2}_{x \text{ buah } 0} + 1\underbrace{000...0000_2}_{y \text{ buah } 0} = 1\underbrace{00000...0000}_{x-y-1 \text{ buah } 0}1\underbrace{00000...00_2}_{y \text{ buah } 0} \neq 1\underbrace{000...00_2}_{t \text{ buah } 0}$$
    untuk suatu bilangan bulat non-negatif $t$.
    
\end{lemma*}
Oleh karena itu, persamaan di soal menjadi
\begin{align*}
2^a+2^b+2^c &= 2^{2d}\\
1\underbrace{000...00_2}_{a \text{ buah } 0} + 1\underbrace{000...00_2}_{b \text{ buah } 0} + 1\underbrace{000...00_2}_{c \text{ buah } 0} &= 1\underbrace{000...00_2}_{2d \text{ buah } 0}
\end{align*}
Karena $(a,b,c)$ simetris (ditukar-tukar tetap solusi), maka WLOG $a \ge b \ge c$. Dari lemma di atas, haruslah $b=c$ sehingga
\begin{align*}
1\underbrace{000...00_2}_{a \text{ buah } 0} + 1\underbrace{000...00_2}_{b+1 \text{ buah } 0} &= 1\underbrace{000...00_2}_{2d \text{ buah } 0}
\end{align*}
yang mengharuskan $a=b+1$ sehingga
\begin{align*}
1\underbrace{000...00_2}_{a+1 \text{ buah } 0} &= 1\underbrace{000...00_2}_{2d \text{ buah } 0}.
\end{align*}
Didapat $a=b+1=c+1$ dan $a+1=2d$.
Berarti solusi yang mungkin adalah $(a,b,c,d)=(2d-1, 2d-2, 2d-2, d)$. Dari sini, solusi terbesar $d$ bisa didapat dengan
\begin{align*}
    a+b+c+d \le 500 \implies (2d-1)+(2d-2)+ (2d-2) + d \le 500 \implies 7d - 5 \le 500\\ \implies d \le 72.
\end{align*}
Nilai terbesar yang mungkin dari $d$ adalah \boxed{72} yang dicapai saat\\ $(a,b,c,d)=(143,142,142,72)$.


