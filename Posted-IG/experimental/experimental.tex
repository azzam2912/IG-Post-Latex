\documentclass[12pt]{article}
\usepackage{amsmath}
\usepackage{amssymb}
\usepackage{amsthm}
\usepackage{geometry}

\geometry{a4paper, margin=1in}
\setlength{\parindent}{0pt}

\newtheoremstyle{mystyle}
  {} % Space above
  {} % Space below
  {\itshape} % Body font
  {} % Indent amount
  {\bfseries} % Theorem head font
  {.} % Punctuation after theorem head
  {.5em} % Space after theorem head
  {} % Theorem head spec (can be left empty, meaning `normal')
\theoremstyle{mystyle}
\newtheorem{problem}{Soal}
\newtheorem{lemma}{Lemma}

\begin{document}

\begin{problem}[CMO 2013]
Tentukan semua polinomial $P(x)$ dengan koefisien real sedemikian sehingga
\[ (x+1)P(x-1) - (x-1)P(x) \]
adalah polinomial konstan.
\end{problem}

\section*{Solusi}

Misalkan $T(x) = (x+1)P(x-1) - (x-1)P(x)$. Dari soal diketahui bahwa $T(x)=c$ untuk suatu konstanta real $c$.

\subsubsection*{Kasus 1: $P(x)$ adalah polinomial konstan}
Jika $P(x) = k$ untuk suatu konstanta $k$, maka:
\[ T(x) = (x+1)k - (x-1)k = kx + k - kx + k = 2k \]
Karena $2k$ adalah sebuah konstanta, maka semua polinomial konstan $P(x)=k$ memenuhi syarat.

\subsubsection*{Kasus 2: $P(x)$ bukan polinomial konstan}
Karena $T(x)=c$ adalah konstan, nilainya sama untuk semua $x$. Substitusi beberapa nilai $x$ berikut:
\begin{itemize}
    \item Untuk $x=1$: \quad $T(1) = (1+1)P(1-1) - (1-1)P(1) = 2P(0)$.
    \item Untuk $x=-1$: \quad $T(-1) = (-1+1)P(-1-1) - (-1-1)P(-1) = 2P(-1)$.
\end{itemize}
Karena $T(1) = T(-1) = c$, maka kita dapatkan $2P(0) = 2P(-1)$, yang berarti $P(0) = P(-1) = k$ untuk suatu konstanta $k$.

Sekarang, konstruksi polinomial baru, $Q(x) = P(x) - k$.
Perhatikan bahwa $Q(0) = P(0) - k = 0$ dan $Q(-1) = P(-1) - k = 0$.
Karena $Q(x)$ memiliki akar di $x=0$ dan $x=-1$, maka $Q(x)$ dapat ditulis dalam bentuk:
\[ Q(x) = A(x) \cdot x(x+1) \]
untuk suatu polinomial $A(x)$. Dengan demikian, kita dapat menyatakan $P(x)$ sebagai:
\[ P(x) = Q(x) + k = A(x)x(x+1) + k \]
Substitusikan kembali bentuk $P(x)$ ini ke dalam persamaan awal untuk $T(x)$:
\begin{align*}
    c &= (x+1)P(x-1) - (x-1)P(x) \\
    c &= (x+1)\left[ A(x-1)(x-1)x + k \right] - (x-1)\left[ A(x)x(x+1) + k \right] \\
    c &= x(x-1)(x+1)A(x-1) + k(x+1) - x(x-1)(x+1)A(x) - k(x-1) \\
    c &= x(x^2-1) \left[ A(x-1) - A(x) \right] + kx + k - kx + k \\
    c &= (x^3-x) \left[ A(x-1) - A(x) \right] + 2k
\end{align*}
Agar $T(x)=c$ bisa menjadi konstan, maka suku yang mengandung $x$ haruslah nol atau konstan yang sesuai.
\[ (x^3-x) \left[ A(x-1) - A(x) \right] = c - 2k \]
dengan sisi kanan, $c-2k$, adalah sebuah konstanta. Misalkan $d$ adalah derajat dari $A(x)$.

\begin{lemma}[Well Known Lemma]
Untuk sebarang polinomial $P(x)$ berderajat $k \ge 1$ dan konstanta $a$, polinomial $P(x+a)-P(x)$ adalah polinomial berderajat $k-1$.
\end{lemma}

Jika $d \ge 1$, maka berdasarkan lemma selisih polinomial (lihat di bawah), derajat dari $A(x-1)-A(x)$ adalah $d-1$. Dengan demikian, derajat dari sisi kiri adalah $3 + (d-1) = d+2$. Karena $d \ge 1$, maka $d+2 \ge 3$. Ini berarti sisi kiri adalah polinomial non-konstan, yang menghasilkan kontradiksi.
 
Oleh karena itu, satu-satunya kemungkinan adalah $d < 1$, yang berarti $d=0$ berarti $A(x)=b$ untuk suatu konstanta real $b$.

Dengan demikian, bentuk $P(x)$ adalah:
\[ P(x) = b \cdot x(x+1) + k = bx^2 + bx + k \]

Cek ke soal, polinomial tersebut memenuhi.

Dapat disimpulkan bahwa seluruh polinomial
\[ P(x) = bx^2 + bx + k \]
memenuhi dimana $b$ dan $k$ adalah konstanta real sembarang. (Saat $b=0$, $P(x)$ adalah polinomial konstan).

\end{document}