\section{Soal - Singapore MO 2025}
Pada segitiga $ABC$ dengan $\angle B>90^\circ$, lingkaran dalamnya menyinggung sisi-sisi $BC$ dan $CA$ berturut-turut di $D$ dan $E$. Garis $ED$ dan $AB$ berpotongan di $P$. Lingkaran dalam dari segitiga $AEP$ menyinggung sisi-sisi $PE$ dan $AP$ berturut-turut di $D_1$ dan $E_1$. Garis $E_1D_1$ dan $AE$ berpotongan di $P_1$. Misalkan $P,C,E,B$ terletak pada satu lingkaran (konsiklik). Buktikan bahwa $BE$ sejajar dengan $PP_1$.
\begin{center}

\begin{tikzpicture}[scale=1.5]
    % 1. Define the main triangle ABC with an obtuse angle at B
    \tkzDefPoint(-0.65, 2.5){A}
    \tkzDefPoint(0,0){B}
    \tkzDefPoint(4,0){C}

    % 2. Construct the incircle of triangle ABC
    \tkzInCenter(A,B,C)\tkzGetPoint{I}
    \tkzDefPointBy[projection=onto B--C](I)\tkzGetPoint{D} % Tangency point on BC
    \tkzDefPointBy[projection=onto C--A](I)\tkzGetPoint{E} % Tangency point on CA
    \tkzDrawCircle[cyan](I,D)

    % 3. Find point P as the intersection of lines ED and AB
    \tkzInterLL(E,D)(A,B)\tkzGetPoint{P}

    % 4. Construct the incircle of triangle AEP
    \tkzInCenter(A,E,P)\tkzGetPoint{I1}
    \tkzDefPointBy[projection=onto P--E](I1)\tkzGetPoint{D1} % Tangency point on PE
    \tkzDefPointBy[projection=onto A--P](I1)\tkzGetPoint{E1} % Tangency point on AP
    \tkzDrawCircle[orange](I1,D1)

    % 5. Find point P1 as the intersection of lines E1D1 and AE
    \tkzInterLL(E1,D1)(A,E)\tkzGetPoint{P1}
    
    % 6. Draw the circle passing through P, C, E, B
    \tkzCircumCenter(P,C,E)\tkzGetPoint{O_circ}
    \tkzDrawCircle[gray](O_circ,P)

    % 7. Draw the main geometric figures
    \tkzDrawPolygon[thick](A,B,C)
    \tkzDrawSegment[thin](E,P)
    \tkzDrawSegment[thin](E1,P1)
    \tkzDrawSegment[thick](B,P)
    
    % 8. Draw the final lines to be proven parallel
    \tkzDrawSegment[blue, thick](B,E)
    \tkzDrawSegment[blue, thick](P,P1)

    % 9. Draw and label all points
    \tkzDrawPoints[fill=black, size=3](A,B,C,D,E,P,D1,E1,P1)
    \tkzLabelPoints[below left](A,B,P,E1)
    \tkzLabelPoints[below right](C)
    \tkzLabelPoints[below right](D, D1)
    \tkzLabelPoints[right](E, P1)
    %\tkzLabelPoints[above right](I, I1)

\end{tikzpicture}

\end{center}


\section{Solusi}
\begin{center}

\begin{tikzpicture}[scale=1.2]
    % 1. Define the main triangle ABC with an obtuse angle at B
    \tkzDefPoint(-0.65, 2.5){A}
    \tkzDefPoint(0,0){B}
    \tkzDefPoint(4,0){C}

    % 2. Construct the incircle of triangle ABC
    \tkzInCenter(A,B,C)\tkzGetPoint{I}
    \tkzDefPointBy[projection=onto B--C](I)\tkzGetPoint{D} % Tangency point on BC
    \tkzDefPointBy[projection=onto C--A](I)\tkzGetPoint{E} % Tangency point on CA
    \tkzDrawCircle[cyan](I,D)

    % 3. Find point P as the intersection of lines ED and AB
    \tkzInterLL(E,D)(A,B)\tkzGetPoint{P}

    % 4. Construct the incircle of triangle AEP
    \tkzInCenter(A,E,P)\tkzGetPoint{I1}
    \tkzDefPointBy[projection=onto P--E](I1)\tkzGetPoint{D1} % Tangency point on PE
    \tkzDefPointBy[projection=onto A--P](I1)\tkzGetPoint{E1} % Tangency point on AP
    \tkzDrawCircle[orange](I1,D1)

    % 5. Find point P1 as the intersection of lines E1D1 and AE
    \tkzInterLL(E1,D1)(A,E)\tkzGetPoint{P1}
    
    % 6. Draw the circle passing through P, C, E, B
    \tkzCircumCenter(P,C,E)\tkzGetPoint{O_circ}
    \tkzDrawCircle[gray](O_circ,P)


    % Draw the circle passing through E, E1, P
    \tkzCircumCenter(P,E1,E)\tkzGetPoint{O_circ_1}
    \tkzDrawCircle[gray, dashed](O_circ_1,E)
    
    % 7. Draw the main geometric figures
    \tkzDrawPolygon[thick](A,B,C)
    \tkzDrawSegment[thin](E,P)
    \tkzDrawSegment[thin](E1,P1)
    \tkzDrawSegment[thick](B,P)
    \tkzDrawSegments[red, thick](C,P E,E1)
    
    % 8. Draw the final lines to be proven parallel
    \tkzDrawSegment[blue, thick](B,E)
    \tkzDrawSegment[blue, thick](P,P1)

    % 9. Draw and label all points
    \tkzDrawPoints[fill=black, size=3](A,B,C,D,E,P,D1,E1,P1)
    \tkzLabelPoints[below left](A,B,P,E1)
    \tkzLabelPoints[below right](C)
    \tkzLabelPoints[below right](D, D1)
    \tkzLabelPoints[right](E, P1)
    %\tkzLabelPoints[above right](I, I1)

\end{tikzpicture}

\end{center}
(Notasikan $\dangle$ sebagai sudut berarah)
Dari properti garis singgung lingkaran, jelas bahwa $CD=CE$, dan $PE_1=PD_1$ sehingga $\dangle CDE = \dangle DEC$ dan $\dangle PE_1D_1 = \dangle E_1D_1P$. 
Oleh karena itu, dari fakta $EBPC$ siklis, didapat
\begin{align*}
    \dangle BDP = \dangle CDE = \dangle PEC = \dangle PBC = \dangle PBD \implies BP = PD.
\end{align*}
Karena $\dangle DPB = \dangle D_1PE_1$ maka $\triangle DPB \sim \triangle D_1PE_1$ sehingga $BD \parallel E_1D_1$. Dari fakta ini, diperoleh
\begin{align*}
    \dangle PE_1P_1 = \dangle PBC = \dangle PEC = \dangle PEP_1 \implies PE_1EP_1 \text{ siklis}.
\end{align*}
Sekarang perhatikan karena $\dangle APE = \dangle BCA$ dan $\dangle PAE = \dangle BAC$ 
maka $\triangle PAE \sim \triangle BAC$ maka rasio segmen-segmen yang bersinggungannya juga memiliki rasio yang sama dengan rasio sisi-sisi bersesuaian dari dua segitiga tersebut.
Oleh karena itu, $AE:AE_1 = AC:AP$ sehingga $EE_1 \parallel CP$. Dari fakta ini, diperoleh
\begin{align*}
    \dangle EBE_1 = \dangle EBP = \dangle ECP = \dangle AEE_1 = \dangle P_1EE_1 = \dangle P_1PE_1 \implies BE \parallel PP_1
\end{align*} \qed
