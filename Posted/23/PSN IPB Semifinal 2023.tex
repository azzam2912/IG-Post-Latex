\documentclass[12pt]{scrartcl}
\usepackage[sexy]{evan}
\usepackage{graphicx,amsmath,amssymb,amsthm,amsfonts,babel}
\usepackage{tikz, tkz-euclide}
\usepackage{lipsum}
\usepackage{setspace}
\graphicspath{ {./} }
\usetikzlibrary{calc,through,intersections}
\usepackage[paperwidth=16cm, paperheight=16cm,margin=1cm]{geometry}
%\usepackage[rightcaption]{sidecap}
\usepackage{caption}
\captionsetup[figure]{font=small, labelformat=empty}
\colorlet{EvanRed}{Red!50!Purple}
\definecolor{officegreen}{rgb}{0.0, 0.5, 0.0}

\newcommand{\siku}[4][.2cm]
	{
	\coordinate (tempa) at ($(#3)!#1!(#2)$);
	\coordinate (tempb) at ($(#3)!#1!(#4)$);
	\coordinate (tempc) at ($(tempa)!0.5!(tempb)$);%midpoint
	\draw[black] (tempa) -- ($(#3)!2!(tempc)$) -- (tempb);
	}
	\usetikzlibrary{calc,positioning,intersections}

\setstretch{1.5}

\usepackage{etoolbox}
\newcommand{\zerodisplayskips}{%
  \setlength{\abovedisplayskip}{5pt}%
  \setlength{\belowdisplayskip}{5pt}%
  \setlength{\abovedisplayshortskip}{5pt}%
  \setlength{\belowdisplayshortskip}{5pt}}
\appto{\normalsize}{\zerodisplayskips}
\appto{\small}{\zerodisplayskips}
\appto{\footnotesize}{\zerodisplayskips}
\setlength\parindent{10pt}

\title{Ineq from Semifinal PSN IPB 2023}
\author{Azzam L. H.}
\date{last update: \today}


\begin{document}
\maketitle
\pagestyle{plain}
\section{Soal}
Diberikan $a,b,c \in \RR^+$. Jika $\dfrac{1}{a}+\dfrac{1}{b}+\dfrac{1}{c}=3$, buktikan bahwa
\begin{align*}
    \dfrac{1}{\sqrt{a^3+b}}+\dfrac{1}{\sqrt{b^3+c}}+\dfrac{1}{\sqrt{c^3+a}} \le \dfrac{3}{\sqrt{2}}.
\end{align*}

\newpage
\section{Solusi}
\textit{\textbf{Motivasi: }} Aku kepikiran CS (Cauchy-Schwarz) dan AM-GM karena penyebutnya akar, terus $\sqrt{a^3+b}$ pangkatnya 3/2 dijadiin $\sqrt{a^3b}$  yang derajatnya 4/2=2 habis itu diakarin lagi jadi $\sqrt{a\sqrt{ab}}$ yang jadi derajat 2/2=1 agar $\sum \frac{1}{a} = 3$ nya bisa dipake karena penyebutnya berderajat 1.\\
\textit{\textbf{Solusi:}}
\begin{align*}
    \sum \dfrac{1}{\sqrt{a^3+b}}
    &\overset{AM-GM}{\le}
    \sum\dfrac{1}{\sqrt{2a\sqrt{ab}}}\\
    &=
    \dfrac{1}{\sqrt{2}}\sum\dfrac{1}{\sqrt{a}}\dfrac{1}{\sqrt{\sqrt{{ab}}}}\\
    &\overset{CS}{\le}
    \dfrac{1}{\sqrt{2}}\sqrt{\left(\sum\dfrac{1}{a}\right)\left(\sum\dfrac{1}{\sqrt{ab}}\right)}\\
    &\overset{AM-GM}{\le}
    \dfrac{1}{\sqrt{2}}\sqrt{\left(\sum\dfrac{1}{a}\right)\left(\sum\dfrac{\frac{1}{a}+\frac{1}{b}}{2}\right)}\\
    &=
    \dfrac{1}{\sqrt{2}}\sqrt{\left(\sum\dfrac{1}{a}\right)\left(\sum\dfrac{1}{a}\right)}\\
    &=
    \dfrac{1}{\sqrt{2}}\left(\sum\dfrac{1}{a}\right)
    =
    \dfrac{3}{\sqrt{2}}
\end{align*}
dengan kesamaan terjadi saat $a=b=c=1$. Terbukti. \qed

\end{document}
