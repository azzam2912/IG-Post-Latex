\documentclass[12pt]{scrartcl}
\usepackage[sexy]{evan}
\usepackage{graphicx,amsmath,amssymb,amsthm,amsfonts,babel}
\usepackage{tikz, tkz-euclide}
\usepackage{lipsum}
\usepackage{setspace}
\graphicspath{ {./} }
\usetikzlibrary{calc,through,intersections}
\usepackage[paperwidth=16cm, paperheight=16cm,margin=1cm]{geometry}
%\usepackage[rightcaption]{sidecap}
\usepackage{caption}
\captionsetup[figure]{font=small, labelformat=empty}
\colorlet{EvanRed}{Red!50!Purple}
\definecolor{officegreen}{rgb}{0.0, 0.5, 0.0}


\newcommand{\siku}[4][.2cm]
	{
	\coordinate (tempa) at ($(#3)!#1!(#2)$);
	\coordinate (tempb) at ($(#3)!#1!(#4)$);
	\coordinate (tempc) at ($(tempa)!0.5!(tempb)$);%midpoint
	\draw[black] (tempa) -- ($(#3)!2!(tempc)$) -- (tempb);
	}
	\usetikzlibrary{calc,positioning,intersections}

\setstretch{1.5}

\usepackage{etoolbox}
\newcommand{\zerodisplayskips}{%
  \setlength{\abovedisplayskip}{5pt}%
  \setlength{\belowdisplayskip}{5pt}%
  \setlength{\abovedisplayshortskip}{5pt}%
  \setlength{\belowdisplayshortskip}{5pt}}
\appto{\normalsize}{\zerodisplayskips}
\appto{\small}{\zerodisplayskips}
\appto{\footnotesize}{\zerodisplayskips}
\setlength\parindent{10pt}

\title{Random Geometry Problem}
\author{Azzam, haxuv.world}
\date{last update: \today}


\begin{document}
\maketitle
\pagestyle{plain}
\section{Soal}
Diberikan persegi panjang $ABCD$. Titik $P$ adalah perpotongan garis $BC$ dengan garis yang melalui $A$ dan tegak lurus $AC$. Titik $Q$ pada segmen $CD$. Garis $PQ$ memotong garis $AD$ di $R$. Garis $BR$ memotong garis $AQ$ di $X$. Buktikan bahwa jika titik $Q$ bergerak sepanjang segmen $CD$, maka besar $\angle BXC$ konstan.

\newpage
\section{Solusi}
\begin{proof}[.] 
.
\begin{center}
    \begin{asy}
        import olympiad;
        import geometry;
        size(340);

        pair A,B,C,D,P,Q,R,X,S;
        A = (-4,3);
        B = (4,3);
        C = (4,-3);
        D = (-4,-3);
        line l = perpendicular(A, line(A,C));
        P = intersectionpoint(l, line(B,C));
        Q = C/3+2*D/3;
        R = intersectionpoint(line(P,Q),line(A,D));
        X = intersectionpoint(line(B,R),line(A,Q));
        S = foot(A,P,R);
        dot("$A$", A, NW);
        dot("$B$", B, NE);
        dot("$C$", C, SE);
        dot("$D$", D, SW);
        dot("$P$", P, N);
        dot("$Q$", Q, SE);
        dot("$Q'$", Q, NW);
        dot("$R$", R, NW);
        dot("$X$", X, SW);
        dot("$X'$", X, SE);
        dot("$S$", S, SE);
        draw(A--C);
        draw(A--P);
        draw(A--S--B);
        draw(B--P--R--A);
        draw(A--X--C);
        draw(rightanglemark(A,X,C,10), red);
        draw(rightanglemark(A,S,Q,10), red);
        draw(rightanglemark(A,D,C,10), red);
        draw(rightanglemark(A,B,P,10), red);
        draw(B--R);
        draw(A--B--C--D--A);
        draw(circumcircle(triangle(A,B,C)), red);
        draw(circumcircle(triangle(A,B,S)), blue);
        draw(circumcircle(triangle(A,D,Q)));
        draw(circumcircle(triangle(B,S,Q)), dashed+blue);
        draw(B--S--Q--X--B, orange);
    \end{asy}
\end{center}
Akan dibuktikan bahwa $ABCXD$ siklis. 

Misalkan $X'$ adalah perpotongan $BR$ dengan lingkaran $(ABCD)$. Misalkan pula $Q'$ perpotongan $AX'$ dengan $CD$.

Definisikan $S$ sebagai proyeksi $A$ ke $PQ$. Karena $\angle ASQ = \angle ADQ = 90^\circ$ maka $ASQD$ siklis.

Berarti dengan \textit{power of a point}
$$RX'\cdot RB = RD\cdot DA = RQ\cdot QS.$$
yang menunjukkan bahwa $X'BSQ$ siklis. Berarti $\angle QX'B = \angle PSB$.

Sadari bahwa karena $\angle ABP = \angle PAC$ dan $\angle BCA = \angle PCA$ maka $\triangle BAP \sim \triangle BCA$ yang langsung mengakibatkan $\angle PAB = \angle ACB$.
Sekarang karena $\angle ASP = \angle ABP = 90^\circ$ maka $ASBP$ siklis, sehingga didapat $$\angle PSB = \angle PAB = \angle ACB = \angle AX'B = \angle Q'X'B.$$

Dari kedua fakta tersebut didapat $\angle QX'B = \angle Q'X'B$ yang menunjukkan $Q = Q'$ yang mengakibatkan $X = X'$. Soal terbukti.
\end{proof}

\end{document}
