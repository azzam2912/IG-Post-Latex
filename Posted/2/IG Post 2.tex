\documentclass{article}
\usepackage[a4paper, total={4in, 8in}]{geometry}
\usepackage[document]{ragged2e}
\addtolength{\topmargin}{1in}
\addtolength{\textheight}{2in} 
\usepackage{amsmath, amsthm}
\usepackage{amssymb}
\usepackage{ulsy}

\setlength{\parindent}{4em}
\setlength{\parskip}{1em}
\renewcommand{\baselinestretch}{1.3}

\usepackage{collectbox}

\makeatletter
\newcommand{\sqbox}{%
	\collectbox{%
		\@tempdima=\dimexpr\width-\totalheight\relax
		\ifdim\@tempdima<\z@
		\fbox{\hbox{\hspace{-.5\@tempdima}\BOXCONTENT\hspace{-.5\@tempdima}}}%
		\else
		\ht\collectedbox=\dimexpr\ht\collectedbox+.2\@tempdima\relax
		\dp\collectedbox=\dimexpr\dp\collectedbox+.5\@tempdima\relax
		\fbox{\BOXCONTENT}%
		\fi
	}%
}
\makeatother


\begin{document}
	
	Sebuah operasi biner $*$ memenuhi $$a*(b*c) = (a*b)\times c \text{  dan  } a*a = 1$$
	untuk seluruh bilangan real non-negatif $a$, $b$, dan $c$ (dengan $\times$ menotasikan perkalian antar dua bilangan real). Tentukan nilai $x$ yang memenuhi $$2016*(6*x)=100.$$\\
	\newpage
	$\textbf{Solusi.}$ Substitusi $c$ dan $b$ dengan $a$ pada persamaan di soal, kita punya
			$$a*1 = a*(a*a)=(a*a)\times a = 1 \times a = a$$
			Selanjutnya substitusi $c$ dengan $b$ pada soal dan kalikan dengan $\frac{1}{b}$ didapat
			$$ a*b= \frac{1}{b}\times((a*b)\times b) = \frac{1}{b}\times(a*(b*b))= \frac{1}{b}\times(a*1)=\frac{1}{b}\times a=\frac{a}{b}$$
			Berarti kita dapat
			$$100 = 2016*(6*x) = \dfrac{2016}{\dfrac{6}{x}}=\frac{2016x}{6} \implies \sqbox{$ x= \dfrac{25}{84}$} \qed$$
	
	
\end{document}