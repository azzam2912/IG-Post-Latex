\documentclass[12pt]{scrartcl}
\usepackage[sexy]{evan}
\usepackage{graphicx,amsmath,amssymb,amsthm,amsfonts,babel}
\usepackage{tikz, tkz-euclide}
\usepackage{lipsum}
\usepackage{setspace}
\graphicspath{ {./} }
\usetikzlibrary{calc,through,intersections}
\usepackage[paperwidth=16cm, paperheight=16cm,margin=1cm]{geometry}

\colorlet{EvanRed}{Red!50!Purple}

\newcommand{\siku}[4][.5cm]
	{
	\coordinate (tempa) at ($(#3)!#1!(#2)$);
	\coordinate (tempb) at ($(#3)!#1!(#4)$);
	\coordinate (tempc) at ($(tempa)!0.5!(tempb)$);%midpoint
	\draw[black] (tempa) -- ($(#3)!2!(tempc)$) -- (tempb);
	}
	\usetikzlibrary{calc,positioning,intersections}

\setstretch{1.5}

\usepackage{etoolbox}
\newcommand{\zerodisplayskips}{%
  \setlength{\abovedisplayskip}{5pt}%
  \setlength{\belowdisplayskip}{5pt}%
  \setlength{\abovedisplayshortskip}{5pt}%
  \setlength{\belowdisplayshortskip}{5pt}}
\appto{\normalsize}{\zerodisplayskips}
\appto{\small}{\zerodisplayskips}
\appto{\footnotesize}{\zerodisplayskips}
\setlength\parindent{10pt}

\title{Suatu Soal Ketaksamaan Sederhana}
\author{}
\date{}


\begin{document}
\maketitle
\pagestyle{plain}
\vspace{-1.5cm}
\section{Soal}
Misalkan $a,b,c > 0$ dan memenuhi $a+b+c=1$. Tunjukkan bahwa
\begin{align*}
\dfrac{ab}{1+c}+\dfrac{bc}{1+a}+\dfrac{ca}{1+b} \ge \dfrac{1}{4}.
\end{align*}

\newpage
\section{Observasi}
Jika dilihat pada setiap suku, penyebut dan pembilang memiliki "derajat" yang berbeda, dengan pembilang memiliki derajat 2 dan penyebut memiliki derajat 1. Contohnya pada $\frac{ab}{1+c}$, ekspresi $ab$ memiliki derajat 2, dan ekspresi $1+c$ memiliki derajat 1. Oleh karena

\end{document}