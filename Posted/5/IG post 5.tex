\documentclass{extarticle}
\usepackage[utf8]{inputenc}
\usepackage{amsmath}
\usepackage{amssymb}
\usepackage{graphicx}
\usepackage[vcentering,dvips]{geometry}
\geometry{papersize={12cm,12cm},total={10cm,10.5cm}}
%\graphicspath{{./Rigan/}}
\renewcommand{\baselinestretch}{1.5}
%\addtolength{\oddsidemargin}{1in}
%\addtolength{\evensidemargin}{1.5in}
%\addtolength{\textwidth}{1.9in}

\addtolength{\topmargin}{0in}
\addtolength{\textheight}{-0.5in} 

\setlength{\parindent}{0em}

\title{Soal Seleksi KNMIPA 2021 Tingkat UI}
\date{}
\begin{document}
	\maketitle

	%\begin{enumerate}
		 \textbf{Soal. }Diketahui $G$ adalah grup dengan elemen identitas $e$.\\ Diketahui juga untuk setiap $a,b,c \in G \backslash \{e\}$ berlaku $abc=cba$. Buktikan bahwa $G$ abelian.
	%\end{enumerate}
	
	\newpage
	
	\textbf{Soal. }Diketahui $G$ adalah grup dengan elemen identitas $e$.\\ Diketahui juga untuk setiap $a,b,c \in G \backslash \{e\}$ berlaku $abc=cba$. Buktikan bahwa $G$ abelian.\\
	
	\textbf{Solusi. } Misalkan $x,y \in G$. Misalkan $x^{-1},y^{-1}\in G$ adalah invers dari $x$ dan $y$ sehingga $xx^{-1} = x^{-1}x=e$ dan $yy^{-1}=y^{-1}y=e$ Misalkan pula $e$ adalah elemen identitas di $G$.
	 \begin{itemize}
	 	\item Jika salah satu dari $x$ atau $y$ sama dengan $e$, maka jelas dari properti grup bahwa $xy = yx$ karena $xy = xe = ex = yx$ (untuk $y=e$) atau $xy = ey = ye = yx$ (untuk $x=e$). 
	 	\item Jika $x \neq e$ dan $y \neq e$, kita punya dua kasus:
	 	\begin{itemize}
	 		\item Kasus I. $xy = e$\\ Maka $yx = yxe = yxyy^{-1} = yey^{-1} = yy^{-1} = e = xy$, berarti $xy = yx$.
	 		\item Kasus II. $xy \neq e$\\Pada $abc = cba$ untuk setiap $a,b,c \in G\backslash\{e\}$, ganti $a$ dengan $xc^{-1}$, $b$ dengan $c$, $c$ dengan $y$ Kita punya \begin{align*}
	 			abc &= cba\\
	 			xc^{-1}cy &= ycxc^{-1}\\
	 			xey &= ycxc^{-1}\\
	 			xy &= ycxc^{-1}.
	 		\end{align*}
 		Selanjutnya ganti $c$ dengan $y^{-1}$, maka
 		\begin{align*}
 			xy &= yy^{-1}x(y^{-1})^{-1}\\
 			xy &= exy\\
 			xy &= yx
 		\end{align*}
	 	\end{itemize}
	 \end{itemize}
	Karena $xy = yx$ untuk sembarang $x,y\in G$, maka berlaku sifat komutatif di $G$ yang berarti membuktikan bahwa $G$ abelian. $\square$
\end{document}
