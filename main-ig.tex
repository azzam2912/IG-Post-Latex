\documentclass[12pt]{scrartcl}
\usepackage[hagavi]{azzam}
\usepackage[paperwidth=17cm, paperheight=17cm,margin=0.6cm]{geometry}
% 4:5, 17:21,25
%\usepackage[rightcaption]{sidecap}
\usepackage{caption}
\captionsetup[figure]{font=small, labelformat=empty}
\colorlet{EvanRed}{Red!50!Purple}
\definecolor{officegreen}{rgb}{0.0, 0.5, 0.0}

\newcommand{\siku}[4][.2cm]
	{
	\coordinate (tempa) at ($(#3)!#1!(#2)$);
	\coordinate (tempb) at ($(#3)!#1!(#4)$);
	\coordinate (tempc) at ($(tempa)!0.5!(tempb)$);%midpoint
	\draw[black] (tempa) -- ($(#3)!2!(tempc)$) -- (tempb);
	}
	\usetikzlibrary{calc,positioning,intersections}


\usepackage{etoolbox}
\newcommand{\zerodisplayskips}{%
  \setlength{\abovedisplayskip}{5pt}%
  \setlength{\belowdisplayskip}{5pt}%
  \setlength{\abovedisplayshortskip}{5pt}%
  \setlength{\belowdisplayshortskip}{5pt}}
\appto{\normalsize}{\zerodisplayskips}
\appto{\small}{\zerodisplayskips}
\appto{\footnotesize}{\zerodisplayskips}
\setlength\parindent{10pt}

% %japanese
% \usepackage{CJKutf8}
% \begin{CJK*}{UTF8}{min}
% \title{簡単な不等式 - 償還}
% \end{CJK*}

\title{Soal NT yang beneran bisa bikin NT}
\author{\small By Azzam L.H. IG: haxuv.world}

% \author{Azzam L. H.}
% \begin{CJK*}{UTF8}{min}
\begin{document}
\maketitle
\pagestyle{plain}
\vspace{-1.5cm}
\section{Soal - Singapore MO 2025}
Untuk sembarang bilangan bulat positif 4-angka $n$, definisikan $f(n)=(a+b)^2$, dengan $a$ adalah bilangan yang dibentuk oleh dua digit pertama dan $b$ adalah bilangan yang dibentuk oleh dua digit terakhir dari $n$ (angka nol di depan diperbolehkan).

Sebagai contoh,
$$f(2025) = (20+25)^2 = 45^2 = 2025.$$

Tentukan semua bilangan bulat positif 4-angka $n$ yang memenuhi $f(n)=n$.

\newpage
\section{Solusi}
\begin{remark*}
    \begin{itemize}
        \item Motivasi pertama: coba-coba...
        \item Stuck? Ke motivasi kedua: perhatikan sifat bilangan yang digit ke berapanya diketahui. Contohnya: $523 = 5 \times 10^2 + 2 \times 10^1 + 3 \times 10^0$. Coba pikirin kalo $\overline{ab}$ jadi apa...
        \item Masih stuck? Gini, biasanya soal-soal NT ngga jauh dari sifat keterbagian (kalo ada persamaan ya liat kalo kiri terbagi sesuatu, kanan juga), modulo...
        \item Beneran masih stuck? Yaudah, liat solusi, hehehe.
    \end{itemize}
\end{remark*}
Akan dicari $a$ dan $b$ yang memenuhi
\begin{align*}
    (a+b)^2&=100a+b\\
    (a+b)^2 - (a+b) &= 99a\\
    (a+b)(a+b-1) &= 99a
\end{align*}
Misalkan $a = dx$ dan $b = dy$ dengan $d,x,y \in \ZZ^+$  yang memenuhi $FPB(x,y) = 1$. Maka
\begin{align*}
    (dx+dy)(dx+dy-1) &= 99dx\\
    (x+y)((x+y)d-1) &= 99x.
\end{align*}
Selanjutnya, karena $FPB(x,x+y)=1$ maka $x+y \nmid x$ sehingga $x+y \mid 99$. Kita juga punya $x+y \ge 1+1 = 2$. Berarti akan dibagi kasus untuk $x+y=3,9,11,33,99$.
\begin{enumerate}[(I)]
    \item Jika $x+y = 3$ maka $3(3d-1) = 99x \implies 3d-1 = 33x$, kontradiksi karena sisi kiri tidak terbagi 3 sedangkan sisi kanan terbagi 3.

    \item Jika $x+y= 9$ maka $9(9d-1) = 99x \implies 9d = 11x+1$ sehingga $11x+1 \equiv 0 \mod 9 \implies 2x \equiv 8 \mod 9 \implies x \equiv 4 \mod 9$. Lalu, karena $FPB(x,y) = 1$, maka kandidat $(x,y)$ yang mungkin adalah $(4,5)$ dengan. Cek ke persamaan, solusi ini memenuhi dengan $d = 5$ dan $(a,b)=(20,25)$.

    \item Jika $x+y= 11$ maka $11(11d-1) = 99x \implies 11d = 9x+1$ sehingga $9x+1 \equiv 0 \mod 11 \implies -2x \equiv 10 \mod 11 \implies x \equiv -5 \equiv 6 \mod 11$. Lalu, karena $FPB(x,y) = 1$, maka kandidat $(x,y)$ yang mungkin adalah $(6,5)$ dengan. Cek ke persamaan, solusi ini memenuhi dengan $d = 5$ dan $(a,b)=(30,25)$.

    \item Jika $x+y = 33$ maka $33(33d-1) = 99x \implies 33d-1 = 3x$, kontradiksi karena sisi kiri tidak terbagi 3 sedangkan sisi kanan terbagi 3.

    \item Jika $x+y= 99$ maka $99(99d-1) = 99x \implies 99d-1 = x$. Karena $99d = x+1 \le x+y = 99$ maka $d \le 1 \implies d=1$. Dari sini didapat $x=98$ dan $y=1$. Cek ke persamaan, $(a,b)=(x,y)=(98,1)$ memenuhi.
\end{enumerate}
Dapat disimpulkan bahwa pasangan solusi $(a,b)$ yang memenuhi adalah $(20,25),(30,25),(98,1)$.
\end{document}
% \end{CJK*}

